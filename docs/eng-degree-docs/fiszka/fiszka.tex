\documentclass[12pt,oneside]{book}
\usepackage[
    a4paper, 
    total={6in, 8in}, 
    top=2cm,
    left=3cm,
    right=3cm,
    bottom=1.25cm,
    headheight=1.5cm,
    includeheadfoot
]{geometry}
\usepackage[T1]{polski}
\usepackage{graphicx}
\usepackage{fancyhdr}
\usepackage{mathptmx}
\usepackage{enumitem}
\usepackage[utf8]{inputenc}
\usepackage{import}
\usepackage{titlesec}
\usepackage{multirow}
\usepackage{longtable}
\usepackage{enumitem}

\import{../}{commands.tex}

\fancypagestyle{plain}{
    \fancyhf{}
    \renewcommand{\headrulewidth}{0pt}
    \fancyhead[C]{\uniimage}
    \fancyfoot[C]{\thepage}
}

\renewcommand\thesection{\Alph{chapter}\arabic{section}}

\linespread{1.5}

\begin{document}

\titleformat
    {\chapter} %/{〈command 〉}
    [block] %/[〈shape〉]
    {\bfseries\large} %/ {〈format〉}
    {} %/ {〈label 〉}
    {0ex} %/ {〈sep〉}
    {
        \centering
    } %/ {〈before-code〉}
    {
        \vspace{0ex}%
    } %/ {〈after-code〉}

\titleformat{\section}[hang]{
    \normalfont
    \bfseries
}{\thesection.}{0.5em}{}

\titleformat{\subsection}[hang]{
    \normalfont
    \bfseries
}{\thesubsection.}{0.5em}{}

\titleformat{\chapter}[hang]
  {
    \normalfont
    \bfseries
    \large
    \centering
    \MakeUppercase
}{}{10pt}{}

\titlespacing{\chapter}{5pt}{20pt}{20pt}[5pt]

{{\chapter
    {\MakeUppercase{\underline{\LARGE{Fiszka projektu dyplomowego}}}}
}}

\noindent
\textbf{\large{Organizator}}
\bigskip

\noindent
\begin{tabular}{ |p{5cm}|p{9cm}|}
    \hline
    Nazwa instytucji & \textbf{\MakeUppercase{Wyższa Szkoła Bankowa w Poznaniu} Wydział Finansów i Bankowości} \\
    \hline
    Adres & ul. Powstańców Wielkopolskich 5, 61-895 Poznań \\
    \hline
\end{tabular}

{\let\clearpage\relax
    \chapter{\MakeUppercase{Dane partnerów}}
}

\section{Dane Promotora}

\begin{tabular}{ |p{5cm}|p{9cm}|}
    \hline
    Imię i nazwisko & Izabela Janicka-Lipska \\
    \hline
    Stopień / Tytuł naukowy & dr inż. \\
    \hline
    Data i podpis &  \\ \hline
\end{tabular}


\pagestyle{plain}

\section{Dane członków zespołu projektu}

\membersTable

{\let\clearpage\relax
    \chapter{Założenia projektu}
    }

\section{Opis projektu}

\subsection{Uzasadnienie wyboru tematu}

Problem utylizacji odpadów stał się w ostatnich latach jednym z największych wyzwań dla społeczeństwa i środowiska naturalnego. Śmieci są produkowane w ogromnych ilościach, a nieprawidłowe postępowanie z nimi prowadzi do skażenia powietrza, wody i gleby. W związku z tym, istnieje potrzeba opracowania rozwiązań, które ułatwią zarządzanie odpadami w sposób bardziej skuteczny i zrównoważony.

\textbf{Dzięki rozwojowi technologii uczenia maszynowego, pojawiła się możliwość stworzenia aplikacji mobilnej, która rozpoznawać będzie pojemniki na odpady na podstawie zdjęć oraz oznaczać je na mapie, co ułatwi lokalizację pojemników na dany typ odpadów, co ułatwi proces własciwego utylizowania śmieci dla użytkowników}. Taka aplikacja może również zwiększyć świadomość społeczną w zakresie utylizacji odpadów i przyczynić się do zmniejszenia ilości odpadów zalegających na składowiskach.


Temat ten jest aktualny i ważny, a jednocześnie daje wiele możliwości na zastosowanie różnych technologii i algorytmów uczenia maszynowego. Przy realizacji projektu można wykorzystać m.in. sieci neuronowe, algorytmy uczenia głębokiego, przetwarzanie obrazów oraz uczenie maszynowe w chmurze oraz usługi geolokalizacyjne.

Podsumowując, projekt \topic  jest uzasadniony ze względu na aktualność i ważność tematu, możliwość wykorzystania najnowszych technologii oraz potencjalne korzyści dla społeczeństwa i środowiska naturalnego.

\subsection{Problem badawczy}
Algorytmy uczenia maszynowego najlepiej nadające się do rozpoznawania pojemników na odpady na podstawie zdjęć i cechy obrazów odpowiadające za skuteczność procesu rozpoznawania



W ramach tego problemu badawczego można skoncentrować się na kilku podproblemach, m.in.:

    -- Analiza dostępnych zbiorów danych: znalezienie zbiorów danych do treningu i testowania modelu rozpoznawania pojemników na odpady, określenie jak są one zdefiniowane oraz jakie informacje zawierają

    -- Wybór odpowiednich algorytmów uczenia maszynowego: na podstawie analizy zbiorów danych, określenie algorytmów uczenia maszynowego, które najlepiej będą się nadawały do rozpoznawania pojemników na odpady na podstawie zdjęć

    -- Przygotowanie zbiorów danych: zastosowanie technik przetwarzania obrazów, aby zbiory danych były odpowiednio przygotowane do treningu modelu oraz potencjalne użycie technik augmentacji danych

    -- Implementacja i trening modelu: wyłonić najlepsze techniki trenowania modelu bazowanego na obrazach

    -- Ocena skuteczności modelu: określenie metryk, które pozwolą dokładnie ocenić skuteczność modelu w rozpoznawaniu pojemników na odpady

    -- Analiza cech obrazów: na podstawie analizy modelu, zlokalizować cechy obrazów odpowiadające za skuteczność procesu rozpoznawania pojemników na odpady

\subsection{Cel główny i cele szczegółowe projektu}

Celem nadrzędnym jest stworzenie aplikacji mobilnej, która będzie pomagać użytkownikom w identyfikacji oraz lokalizowaniu pojemników na odpady danego typu. Aplikacja ta będzie miała na celu wspomaganie procesu utylizacji odpadów w odpowiedzialny i racjonalny sposób poprzez ułatwienie segregacji odpadów oraz zapobieganie ich niewłaściwemu składowaniu.

Cele podrzędne:

    Analiza dostępnych zbiorów danych: analiza zbiorów danych dostępnych w Internecie, które mogą posłużyć do treningu i testowania modelu rozpoznawania pojemników na odpady.

    Wybór odpowiednich algorytmów uczenia maszynowego: wybór algorytmów uczenia maszynowego, które będą najlepiej nadawały się do rozpoznawania pojemników na odpady na podstawie zdjęć.

    Przygotowanie zbiorów danych: przygotowanie zbiorów danych do treningu modelu rozpoznawania pojemników na odpady.

    Implementacja i trening modelu: implementacja i trening modelu rozpoznawania pojemników na odpady.

    Ocena skuteczności modelu: ocena skuteczności modelu rozpoznawania pojemników na odpady.

    Analiza cech obrazów: analiza cech obrazów odpowiadających za skuteczność procesu rozpoznawania pojemników na odpady.

    Określenie, które cechy obrazów mają największy wpływ na skuteczność procesu rozpoznawania i jak można je wykorzystać do dalszej optymalizacji modelu.

\subsection{Zakres podmiotowy, przedmiotowy, czasowy i przestrzenny}

-- Zakres podmiotowy projektu \topic  obejmuje badanie i opracowanie aplikacji mobilnej oraz algorytmów uczenia maszynowego, które umożliwią rozpoznawanie pojemników na odpady. Podmiotem badania jest zatem aplikacja mobilna Trashify oraz modele uczenia maszynowego, które będą umożliwiać rozpoznawanie pojemników na odpady.

-- Zakres przedmiotowy obejmuje badanie możliwości rozpoznawania różnych typów pojemników na odpady (np. pojemnik na papier, szkło, plastik, odpady organiczne itp.) oraz opracowanie algorytmów uczenia maszynowego, które umożliwią ich poprawną identyfikację. W ramach projektu będzie również opracowana aplikacja mobilna, która będzie integrować te algorytmy oraz udostępniać użytkownikom informacje o pojemnikach na odpady oraz umożliwiać im łatwe i szybkie ich zlokalizowanie.

-- Zakres czasowy projektu obejmuje okres od 01.04.2023 aż do 01.12.2023.

-- Zakres przestrzenny projektu obejmuje miejsce, w którym aplikacja Trashify będzie wykorzystywana, czyli przede wszystkim Polska. Projekt ogranicza się do konkretnego obszaru geograficznego, ponieważ algorytmy uczenia maszynowego, które zostaną opracowane w ramach projektu, będą skupiały się na pojemnikach na odpady spotykanych w Polsce. W ramach projektu nie będzie prowadzona analiza pojemników spotykanych w innych krajach.

\subsection{Metody i techniki badawcze}

\begin{enumerate}
    \item Metoda analizy i konstrukcji logicznej
    
    \begin{enumerate}[label=--]
        \item Analiza składników systemu informacyjnego Trashify na składniki cząstkowe (np. interfejs użytkownika, algorytmy uczenia maszynowego, baza danych itp.)
        \item Indywidualna analiza każdego z tych składników.
        Synteza wyników analizy w celu stworzenia spójnego i logicznego systemu informacyjnego
    \end{enumerate}

    \item Metoda statystyczna
    
    \begin{enumerate}[label=--]
        \item Badanie preferencji użytkowników w zakresie korzystania z aplikacji mobilnej Trashify na ograniczonej próbie.
        \item Pozyskanie informacji o średniej ilości odpadów segregowanych przez użytkowników korzystających z aplikacji.
        \item Analiza zależności pomiędzy poszczególnymi cechami użytkowników a ich zachowaniem w kontekście segregacji odpadów.
    \end{enumerate}

    \item Metoda symulacji komputerowej
    
    \begin{enumerate}[label=--]
        \item Wykorzystanie algorytmów uczenia maszynowego do stworzenia modelu przewidującego ilość i rodzaj odpadów, jakie zostaną wyprodukowane w danym miejscu w przyszłości.
        \item Przeprowadzenie symulacji na tym modelu, aby zbadać wpływ różnych czynników (np. zmiany stylu życia mieszkańców, nowych przepisów regulujących sortowanie odpadów) na ilość i rodzaj produkowanych odpadów.
    \end{enumerate}

    \item Metoda heurystyczna
    
       
    \begin{enumerate}[label=--]
        \item Analiza problemów i wyzwań związanych z korzystaniem z aplikacji Trashify i sortowaniem odpadów przez użytkowników.
        \item Odkrywanie nowych rozwiązań i podejść do tych problemów.
        \item Badanie opinii użytkowników i na ich podstawie wprowadzanie ulepszeń w aplikacji.
    \end{enumerate}

\end{enumerate}

\section{Ryzyko związane z realizacją projektu}

\begin{enumerate}[label=--]
    \item Technologiczne ryzyko - związane z wykorzystaniem nowych technologii, które mogą nie działać zgodnie z oczekiwaniami, bądź w trakcie projektowania i implementacji systemu, mogą pojawić się trudności techniczne.
    \item Ryzyko projektowe - związane z niedostatecznymi planami, wyboru niewłaściwej metodyki lub zła interpretacją wymagań.
    \item Ryzyko jakościowe - związane z niedostatecznymi testami, błędami w kodzie, które mogą prowadzić do awarii systemu, co może prowadzić do utraty zaufania klientów.
    \item Ryzyko bezpieczeństwa - związane z atakami cybernetycznymi na system, które mogą prowadzić do utraty danych lub przestojów w działaniu systemu.
\end{enumerate}

\newpage

{\let\clearpage\relax\chapter{Zadania planowane w ramach projektu}}

\section{Zadania w projekcie}

\begin{longtable}{ | p{0.2\textwidth}| p{0.4\textwidth}| p{0.3\textwidth}| }
    \hline
    Cele szczegółowe projektu & Zadania w projekcie oraz termin rozpoczęcia i zakończenia realizacji zadania & Osoby zaangażowane w realizację zadania \\
    \hline
    \multirow[t]{9}{\linewidth}{Cel 1: Przygotować modele uczenia maszynowego}
    &\task{Zadanie 1: Zebranie danych kontekstowych}{1. Kacper Bylicki}{}{}{}
    \cline{2-3}
    &\task{Zadanie 2: Przygotowanie danych treningowych}{1. Kacper Bylicki}{2. Jakub Barczewski}{3. Marek Gerszendorf}{}
    \cline{2-3}
    &\task{Zadanie 3: Dobranie algorytmów uczenia maszynowego}{1. Kacper Bylicki}{}{}{}
    \cline{2-3}
    &\task{Zadanie 4: Zaimplementowanie algorytmów uczenia maszynowego}{1. Kacper Bylicki}{2. Jakub Barczewski}{}{}
    \cline{2-3}
    &\task{Zadanie 5: Przeprowadzenie testów i poprawek}{1. Kacper Bylicki}{}{}{}
    \hline
    \multicolumn{3}{|l|}{}\\
    \hline
\end{longtable}

\newpage

\begin{longtable}{ | p{0.2\textwidth}| p{0.4\textwidth}| p{0.3\textwidth}| }
    \hline
    Cele szczegółowe projektu & Zadania w projekcie oraz termin rozpoczęcia i zakończenia realizacji zadania & Osoby zaangażowane w realizację zadania \\
    \hline
    \multirow[t]{9}{\linewidth}{Cel 2: Przygotowanie aplikacji serwerowej sterującą algorytmami uczenia maszynowego}
    &\task{Zadanie 1: Stworzenie diagramu architektury aplikacji}{1. Jakub Barczewski}{2. Kacper Bylicki}{}{}
    \cline{2-3}
    &\task{Zadanie 2: Zaimplementowanie kanałów komunikacji między aplikacją, a algorytmami uczenia maszynowego}{1. Jakub Barczewski}{2. Kacper Bylicki}{}{}
    \cline{2-3}
    &\task{Zadanie 3: Zaimplementowanie kanały komunikacji między aplikacją, a aplikacją mobilną}{1. Jakub Barczewski}{}{}{}
    \cline{2-3}
    &\task{Zadanie 4: Przeprowadzenie testów i poprawek}{1. Jakub Barczewski}{2. Kacper Bylicki}{3. Marek Gerszendorf}{}
    \cline{2-3}
    &\task{Zadanie 5: Dokonanie optymalizacji aplikacji}{1. Jakub Barczewski}{}{}{}
    \hline
    \multicolumn{3}{|l|}{}\\
    \hline
\end{longtable}

\newpage

\begin{longtable}{ | p{0.2\textwidth}| p{0.4\textwidth}| p{0.3\textwidth}| }
    \hline
    Cele szczegółowe projektu & Zadania w projekcie oraz termin rozpoczęcia i zakończenia realizacji zadania & Osoby zaangażowane w realizację zadania \\
    \hline
    \multirow[t]{9}{\linewidth}{Cel 3: Przygotowanie aplikacji mobilnej}
    &\task{Zadanie 1: Stworzenie diagramu architektury aplikacji}{1. Marek Gerszendorf}{}{}{}
    \cline{2-3}
    &\task{Zadanie 2: Zaprojektowanie warstwy wizualnej na bazie diagramu}{1. Marek Gerszendorf}{2. Kacper Bylicki}{3. Jakub Barczewski}{}
    \cline{2-3}
    &\task{Zadanie 3: Stworzenie prototypu aplikacji}{1. Marek Gerszendorf}{}{}{}
    \cline{2-3}
    &\task{Zadanie 4: Zaimplementowanie kanałów komunikacji między aplikacją, a aplikacją serwerową}{1. Marek Gerszendorf}{2. Kacper Bylicki}{3. Jakub Barczewski}{}
    \cline{2-3}
    &\task{Zadanie 5: Przeprowadznie testów i poprawek}{1. Marek Gerszendorf}{}{}{}
    \hline
    \multicolumn{3}{|l|}{}\\
    \hline
\end{longtable}

\end{document}
