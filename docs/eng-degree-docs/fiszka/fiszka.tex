\documentclass[12pt,oneside]{book}
\usepackage[
    a4paper, 
    total={6in, 8in}, 
    top=2cm,
    left=3cm,
    right=3cm,
    bottom=1.25cm,
    headheight=1.5cm,
    includeheadfoot
]{geometry}
\usepackage[T1]{polski}
\usepackage{graphicx}
\usepackage{fancyhdr}
\usepackage{mathptmx}
\usepackage{enumitem}
\usepackage[utf8]{inputenc}
\usepackage{import}
\usepackage{titlesec}
\usepackage{multirow}
\usepackage{longtable}
\usepackage{tabularray}

\import{../}{commands.tex}

\fancypagestyle{plain}{
    \fancyhf{}
    \renewcommand{\headrulewidth}{0pt}
    \fancyhead[C]{\uniimage}
    \fancyfoot[C]{\thepage}
}

\renewcommand\thesection{\Alph{chapter}\arabic{section}}

\linespread{1.5}

\begin{document}

\titleformat
    {\chapter} %/{〈command 〉}
    [block] %/[〈shape〉]
    {\bfseries\large} %/ {〈format〉}
    {} %/ {〈label 〉}
    {0ex} %/ {〈sep〉}
    {
        \centering
    } %/ {〈before-code〉}
    {
        \vspace{0ex}%
    } %/ {〈after-code〉}

\titleformat{\section}[hang]{
    \normalfont
    \bfseries
}{\thesection.}{0.5em}{}

\titleformat{\subsection}[hang]{
    \normalfont
    \bfseries
}{\thesubsection.}{0.5em}{}

\titleformat{\chapter}[hang]
  {
    \normalfont
    \bfseries
    \large
    \centering
    \MakeUppercase
}{}{10pt}{}

\titlespacing{\chapter}{5pt}{20pt}{20pt}[5pt]

{{\chapter
    {\MakeUppercase{\underline{\LARGE{Fiszka projektu dyplomowego}}}}
}}

\noindent
\textbf{\large{Organizator}}
\bigskip

\noindent
\begin{tabular}{ |p{5cm}|p{9cm}|}
    \hline
    Nazwa instytucji & \textbf{\MakeUppercase{Wyższa Szkoła Bankowa w Poznaniu} Wydział Finansów i Bankowości} \\
    \hline
    Adres & ul. Powstańców Wielkopolskich 5, 61-895 Poznań \\
    \hline
\end{tabular}

{\let\clearpage\relax
    \chapter{\MakeUppercase{Dane partnerów}}
}

\section{Dane Promotora}

\begin{tabular}{ |p{5cm}|p{9cm}|}
    \hline
    Imię i nazwisko & Izabela Janicka-Lipska \\
    \hline
    Stopień / Tytuł naukowy & dr. inż. \\
    \hline
    Data i podpis &  \\ \hline
\end{tabular}


\pagestyle{plain}

\section{Dane członków zespołu projektu}

\membersTable

{\let\clearpage\relax
    \chapter{Założenia projektu}
    }

\section{Opis projektu}

\subsection{Uzasadnienie wyboru tematu}

Po pierwsze, problem utylizacji odpadów stał się w ostatnich latach jednym z największych wyzwań dla społeczeństwa i środowiska naturalnego. Śmieci są produkowane w ogromnych ilościach, a nieprawidłowe postępowanie z nimi prowadzi do skażenia powietrza, wody i gleby. W związku z tym, istnieje potrzeba opracowania rozwiązań, które pomogą w zarządzaniu odpadami w sposób bardziej skuteczny i zrównoważony.

Po drugie, rozwój technologii uczenia maszynowego pozwala na stworzenie aplikacji mobilnej, która może rozpoznawać pojemniki na odpady na podstawie zdjęć, co ułatwi proces segregacji odpadów dla użytkowników. Taka aplikacja może również zwiększyć świadomość społeczną w zakresie utylizacji odpadów i przyczynić się do zmniejszenia ilości odpadów zalegających na składowiskach.


Po trzecie, temat ten jest aktualny i ważny, a jednocześnie daje wiele możliwości na zastosowanie różnych technologii i algorytmów uczenia maszynowego. Przy realizacji projektu można wykorzystać m.in. sieci neuronowe, algorytmy głębokiego uczenia, przetwarzanie obrazów oraz uczenie maszynowe w chmurze.

Podsumowując, projekt "Trashify - rozpoznawanie pojemników na odpady za pomocą algorytmów uczenia maszynowego" jest uzasadniony ze względu na aktualność i ważność tematu, możliwość wykorzystania najnowszych technologii oraz potencjalne korzyści dla społeczeństwa i środowiska naturalnego.

\subsection{Problem badawczy}
Jakie algorytmy uczenia maszynowego najlepiej nadają się do rozpoznawania pojemników na odpady na podstawie zdjęć i jakie cechy obrazów odpowiadają za skuteczność procesu rozpoznawania?

W ramach tego problemu badawczego można skoncentrować się na kilku podproblemach, m.in.:

    Analiza dostępnych zbiorów danych: jakie zbiory danych są dostępne do treningu i testowania modelu rozpoznawania pojemników na odpady, jak są one zdefiniowane i jakie informacje zawierają?

    Wybór odpowiednich algorytmów uczenia maszynowego: na podstawie analizy zbiorów danych, jakie algorytmy uczenia maszynowego będą najlepiej nadawały się do rozpoznawania pojemników na odpady na podstawie zdjęć?

    Przygotowanie zbiorów danych: jakie techniki należy zastosować, aby zbioru danych były odpowiednio przygotowane do treningu modelu? Czy konieczne jest zastosowanie technik augmentacji danych?

    Implementacja i trening modelu: jakie techniki należy zastosować w procesie trenowania modelu, aby uzyskać jak najlepsze wyniki?

    Ocena skuteczności modelu: jakie metryki należy zastosować, aby dokładnie ocenić skuteczność modelu w rozpoznawaniu pojemników na odpady?

    Analiza cech obrazów: na podstawie analizy modelu, jakie cechy obrazów odpowiadają za skuteczność procesu rozpoznawania pojemników na odpady?

\subsection{Cel główny i cele szczegółowe projektu}

Głównym celem projektu "Trashify - rozpoznawanie pojemników na odpady za pomocą algorytmów uczenia maszynowego" jest stworzenie aplikacji mobilnej, która będzie pomagać użytkownikom w identyfikacji pojemników na odpady danego typu. Aplikacja ta ma na celu wspomaganie procesu utylizacji odpadów w odpowiedzialny i racjonalny sposób poprzez ułatwienie segregacji odpadów i zapobieganie ich niewłaściwemu składowaniu.

Aplikacja Trashify będzie działała w następujący sposób: użytkownik zrobi zdjęcie pojemnika na odpady, a następnie za pomocą algorytmów uczenia maszynowego aplikacja będzie rozpoznawać rodzaj odpadów, które należy do niego wrzucić. Aplikacja będzie również wyświetlała mapę, na której użytkownik będzie mógł znaleźć najbliższy pojemnik na odpady danego typu. Dzięki temu, użytkownik będzie miał łatwy dostęp do informacji na temat właściwej utylizacji odpadów, co pozwoli na zmniejszenie ilości odpadów zalegających na składowiskach oraz zwiększenie świadomości ekologicznej w społeczeństwie.

Projekt ten ma również na celu wykorzystanie najnowszych technologii z zakresu uczenia maszynowego i przetwarzania obrazów. Przygotowanie modelu rozpoznawania pojemników na odpady wymaga zastosowania specjalistycznych algorytmów uczenia maszynowego, takich jak sieci neuronowe lub algorytmy głębokiego uczenia. Przetestowanie tych technologii w praktyce pozwoli na lepsze zrozumienie ich zastosowania w dziedzinie rozpoznawania obrazów.

Wprowadzenie aplikacji Trashify do użytku może przynieść wiele korzyści dla środowiska naturalnego, a także dla ludzi, którzy z niej korzystają. Dzięki temu, aplikacja ta może przyczynić się do zwiększenia efektywności i skuteczności procesu utylizacji odpadów oraz propagowania idei ekologicznej w społeczeństwie.

\subsection{Zakres podmiotowy, przedmiotowy, czasowy i przestrzenny}

Zakres podmiotowy projektu "Trashify - rozpoznawanie pojemników na odpady za pomocą algorytmów uczenia maszynowego" obejmuje badanie i opracowanie aplikacji mobilnej oraz algorytmów uczenia maszynowego, które umożliwią rozpoznawanie pojemników na odpady. Podmiotem badania jest zatem aplikacja mobilna Trashify oraz modele uczenia maszynowego, które będą umożliwiać rozpoznawanie pojemników na odpady.

Zakres przedmiotowy obejmuje badanie możliwości rozpoznawania różnych typów pojemników na odpady (np. pojemnik na papier, szkło, plastik, odpady organiczne itp.) oraz opracowanie algorytmów uczenia maszynowego, które umożliwią ich poprawną identyfikację. W ramach projektu będzie również opracowana aplikacja mobilna, która będzie integrować te algorytmy oraz udostępniać użytkownikom informacje o pojemnikach na odpady oraz umożliwiać im łatwe i szybkie ich zlokalizowanie.

Zakres czasowy projektu obejmuje okres od rozpoczęcia prac badawczych nad algorytmami rozpoznawania pojemników na odpady aż do momentu udostępnienia aplikacji Trashify użytkownikom. Czas ten może być zależny od różnych czynników, takich jak dostępność danych treningowych, stopień skomplikowania algorytmów, ilość testów i poprawek, które będą wymagane w trakcie pracy nad aplikacją.

Zakres przestrzenny projektu obejmuje miejsce, w którym aplikacja Trashify będzie wykorzystywana, czyli przede wszystkim kraje, w których istnieją odpowiednie pojemniki na odpady. Projekt nie ogranicza się jednak do konkretnego obszaru geograficznego, ponieważ algorytmy uczenia maszynowego, które zostaną opracowane w ramach projektu, będą mogły być wykorzystywane w innych aplikacjach i systemach, gdzie wymagane jest rozpoznawanie obrazów.

\subsection{Metody i techniki badawcze}

\begin{enumerate}
    \item Metoda analizy i konstrukcji logicznej
    
    Analiza składników systemu informacyjnego Trashify na składniki cząstkowe (np. interfejs użytkownika, algorytmy uczenia maszynowego, baza danych itp.).

    Indywidualna analiza każdego z tych składników.
    Synteza wyników analizy w celu stworzenia spójnego i logicznego systemu informacyjnego.
    \item Metoda statystyczna
    
    Badanie preferencji użytkowników w zakresie korzystania z aplikacji mobilnej Trashify na ograniczonej próbie.

    Pozyskanie informacji o średniej ilości odpadów segregowanych przez użytkowników korzystających z aplikacji.

    Analiza zależności pomiędzy poszczególnymi cechami użytkowników a ich zachowaniem w kontekście segregacji odpadów.    
    \item Metoda symulacji komputerowej
    
    Wykorzystanie algorytmów uczenia maszynowego do stworzenia modelu przewidującego ilość i rodzaj odpadów, jakie zostaną wyprodukowane w danym miejscu w przyszłości.
    
    Przeprowadzenie symulacji na tym modelu, aby zbadać wpływ różnych czynników (np. zmiany stylu życia mieszkańców, nowych przepisów regulujących sortowanie odpadów) na ilość i rodzaj produkowanych odpadów.

    \item Metoda heurystyczna
    
    Analiza problemów i wyzwań związanych z korzystaniem z aplikacji Trashify i sortowaniem odpadów przez użytkowników.

    Odkrywanie nowych rozwiązań i podejść do tych problemów.
    
    Badanie opinii użytkowników i na ich podstawie wprowadzanie ulepszeń w aplikacji.
\end{enumerate}

\section{Ryzyko związane z realizacją projektu}

{\Huge\textbf{Tymczasowe}}

Technologiczne ryzyko - związane z wykorzystaniem nowych technologii, które mogą nie działać zgodnie z oczekiwaniami, bądź w trakcie projektowania i implementacji systemu, mogą pojawić się trudności techniczne.

Ryzyko biznesowe - związane z opóźnieniami w harmonogramie, niskimi dochodami lub brakiem akceptacji produktu przez klientów, co może prowadzić do niepowodzenia projektu.

Ryzyko projektowe - związane z niedostatecznymi planami, wyboru niewłaściwej metodyki lub zła interpretacją wymagań.

Ryzyko jakościowe - związane z niedostatecznymi testami, błędami w kodzie, które mogą prowadzić do awarii systemu, co może prowadzić do utraty zaufania klientów.

Ryzyko bezpieczeństwa - związane z atakami cybernetycznymi na system, które mogą prowadzić do utraty danych lub przestojów w działaniu systemu.

{\let\clearpage\relax\chapter{Zadania planowane w ramach projektu}}

\section{Zadania w projekcie}

\begin{longtable}{ | p{0.3\textwidth}| p{0.4\textwidth}| p{0.2\textwidth}| }
    \hline
    Cele szczegółowe projektu & Zadania w projekcie oraz termin rozpoczęcia i zakończenia realizacji zadania & Osoby zaangażowane w realizację zadania \\
    \objective
        {Cel 1: Przygotować modele uczenia maszynowego}
        {Zadanie 1: Przygotować dane treningowe}
        {Zadanie 2: Dobrać algorytmy uczenia maszynowego}
        {Zadanie 3: Przeprowadzić testy i poprawki}
    \multicolumn{3}{|l|}{}\\
    \multicolumn{3}{|l|}{}\\
    \objective
        {Cel 2: Przygotować aplikację serwerową sterującą algorytmami uczenia maszynowego}
        {Zadanie 1: Stworzyć diagram architektury aplikacji}
        {Zadanie 2: Zaimplementować aplikację na bazie diagramu}
        {Zadanie 3: Przeprowadzić testy i poprawki}
    \multicolumn{3}{|l|}{}\\
    \objective
        {Cel 3: Przygotować aplikację mobilną}
        {Zadanie 1: Stworzyć diagram architektury aplikacji}
        {Zadanie 2: Zaimplementować aplikację na bazie diagramu}
        {Zadanie 3: Przeprowadzić testy i poprawki}
    \multicolumn{3}{|l|}{}\\
    \objective
        {Cel 4: Pójść na piwo}
        {Zadanie 1: Pójść na piwo}
        {Zadanie 2: Przeżyć kaca}
        {Zadanie 3: Przeprowadzić testy i poprawki}
\end{longtable}

\end{document}
